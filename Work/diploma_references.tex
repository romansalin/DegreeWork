\begin{thebibliography}{9}

\bibitem{mitrofanov}
 Митрофанов Ю. И. Анализ сетей массового обслуживания.~-- Саратов: Научная книга, 2005.~-- 175 с.

\bibitem{kleinrock1}
 Клейнрок Л. Теория массового обслуживания / Пер. с англ.~-- М.: Машиностроение, 1979.~-- 432 с.

\bibitem{kleinrock2}
 Клейнрок Л. Вычислительные системы с очередями / Пер. с англ.~-- М.: Мир, 1979.~-- 600 с.

\bibitem{gnedenko}
 Гнеденко Б. В., Коваленко И. Н. Введение в теорию массового обслуживания.~-- М.: Наука, ГРФМЛ, 1966.~-- 432 c.

\bibitem{yao1}
 Yao D. D., Buzacott J. A. Modeling a class of state-dependent routing in flexible manufacturing systems~// Annals of Operations Research.~-- 1985.~-- No. 3.~-- P. 153-167.

\bibitem{yao2}
 Yao D. D., Buzacott J. A. On queueing network as flexible manufacturing systems~// Queueing Systems.~--1986.~-- No 1.~-- P. 5-27.

\bibitem{yao3}
 Yao D. D., and Buzacott J. A. Modeling the performance of flexible manufacturing systems~// International Journal of Production Research.~--1984.~-- Vol. 23.~-- P. 945-955.

\bibitem{reiser}
 Reiser M., Lavenberg S. S. Mean-value analysis of closed multichain queueing networks~// J. of the Association for Computing Machinery.~-- 1980.~-- Vol. 27, No. 2.~-- P. 313-322.

\bibitem{kelly}
 Kelly F. P. Reversibility and stochastic networks.~-- New York: Wiley, 1979.

\end{thebibliography}

