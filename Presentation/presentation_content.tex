\title{Все что вы хотели знать о сферическом коне в вакууме, но боялись спросить}
\author{Олень Северный}
\institute{Научно исследовательский институт физико-матетматических проблем}
\date{Москва, 2010}
% Создание заглавной страницы
\frame{\titlepage}

\begin{frame}{Вот он наш герой}
 Марковский процесс $\overline{\eta}(\tau)$, определенный в пространстве состояний $S$ и управляемый PSQ--маршрутизацией, как определено в~(\ref{eq:2.2}), является обратимым относительно времени и имеет следующую мультипликативную форму стационарного распределения вероятностей:
 \begin{equation}
  \pi(\overline{\eta}) = G^{-1} \prod_{i=0}^L \left[ \prod_{j=1}^{n_i} \nu_i^{-1} (j) \right]
  \left[ \prod_{t=1}^T \prod_{j=1}^{n_{it}} \frac{r_{it} (j - 1 + \delta_{i0})}{j\mu_{it}} \right], \quad \overline{\eta} \in S ,
  \label{eq:2.4}
 \end{equation}
где $\delta_{i0}=1$, если $i=0$, иначе $\delta_{i0}=0$, и $G$~--- нормализующая константа.
\end{frame}
