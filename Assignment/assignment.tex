\documentclass[a4paper,14pt]{extarticle}

\usepackage{cmap} % improved search for russian words in pdf

% Nice cyrillic fonts
\usepackage{pscyr}
\renewcommand{\rmdefault}{ftm} % Times New Roman
%\renewcommand{\sfdefault}{ftx}
%\renewcommand{\ttdefault}{cmttp} % not good ttf

\usepackage[T2A]{fontenc} % Russian support
\DeclareSymbolFont{T2Aletters}{T2A}{cmr}{m}{it}
\usepackage[utf8]{inputenc} % UTF8
\usepackage[english,russian]{babel}

% Provides support for setting the spacing between lines in a document. Package
% options include singlespacing, onehalfspacing, and doublespacing.
\usepackage{setspace}
\onehalfspacing % one half spacing

\usepackage{indentfirst} % indent
\setlength{\parindent}{1.25cm}

% Alternative geometry
\usepackage[top=1.8cm, bottom=1.8cm, left=2.5cm, right=1.5cm, bindingoffset=0cm,
 			headheight=0cm, footskip=1cm, headsep=0cm]{geometry}

\usepackage{soul} % hyphenation for letterspacing, underlining and more

\sloppy % makes TeX less fussy about line breaking

\usepackage{titlesec} % select alternative section titles

\usepackage{float}

\pagestyle{empty}

\begin{document}

\begin{center}
\singlespacing
Министерство образования и науки Российской Федерации\\
\medskip
\mbox{ФЕДЕРАЛЬНОЕ ГОСУДАРСТВЕННОЕ БЮДЖЕТНОЕ ОБРАЗОВАТЕЛЬНОЕ}\\
УЧРЕЖДЕНИЕ ВЫСШЕГО ПРОФЕССИОНАЛЬНОГО ОБРАЗОВАНИЯ\\
<<САРАТОВСКИЙ ГОСУДАРСТВЕННЫЙ УНИВЕРСИТЕТ\\
ИМЕНИ Н.Г. ЧЕРНЫШЕВСКОГО>>
\end{center}

\vspace{0.5cm}
\begin{flushright}
\parbox{6.8cm}{
\raggedright
  Кафедра системного анализа \\ и автоматического управления
}
\end{flushright}

\vspace{1cm}
\begin{center}
\textbf{ЗАДАНИЕ}\\
\textbf{на выпускную квалификационную работу специалиста}
\end{center}

\begin{flushleft}
по специальности 010501 --- прикладная математика и информатика\\
студента 5 курса факультета компьютерных наук и информационных технологий\\
Салина Романа Владимировича
\medskip

Тема работы: <<Исследование производственных систем с маршрутизацией, зависящей от состояния>>
\end{flushleft}

\vfill

\noindent
\begin{flushleft}
Научный руководитель\\
доцент, к.ф.-м.н. \hfill В. И. Долгов\\
\vspace{10mm}
Заведующий кафедрой\\
д.т.н., профессор \hfill Ю. И. Митрофанов
\end{flushleft}

\vfill

\begin{center}
Саратов 2014
\end{center}

\clearpage
\centerline{\textbf{Содержание работы}}

\noindent
\begin{flushleft}
ВВЕДЕНИЕ\\
1~~Основные понятия массового обслуживания\\
\quad 1.1~~Некоторые распределения случайных величин\\
\quad 1.2~~Процесс размножения и гибели\\
\quad 1.3~~Основные параметры и характеристики сетей массового обслуживания\\
2~~Гибкие производственные системы с маршрутизацией, зависящей от состояния\\
\quad 2.1~~Описание модели\\
\quad 2.2~~Решение уравнения равновесия\\
3~~Алгоритм метода анализа производственных систем с маршрутизацией, зависящей от состояния\\
\quad 3.1~~Описание алгоритма\\
\quad 3.2~~Структурная схема алгоритма\\
4~~Описание и назначение программы\\
\quad 4.1~~Список идентификаторов\\
\quad 4.2~~Описание программы\\
\quad 4.3~~Описание и назначение функций\\
5~~Аспекты практического применения\\
ЗАКЛЮЧЕНИЕ\\
СПИСОК ИСПОЛЬЗОВАННЫХ ИСТОЧНИКОВ\\
Приложение A. Код программы
\end{flushleft}
\begin{flushleft}
\textbf{Срок представления работы:}~\makebox[4cm]{\hrulefill}\\
\hspace{6.8cm} число, месяц, год
\medskip\\
Рассмотрено и одобрено на заседании кафедры системного анализа и~автоматического управления\\
Протокол \textnumero~\makebox[1cm]{\hrulefill}~от~\makebox[4cm]{\hrulefill}\\
\hspace{4.5cm} число, месяц, год\\
Секретарь~\makebox[4cm]{\hrulefill}~~~~~\makebox[5cm]{\hrulefill}\\
\hspace{2.7cm} подпись, дата \hspace{1.5cm} инициалы, фамилия
\medskip\\
Дата выдачи задания~\makebox[4cm]{\hrulefill}\\
\hspace{4.8cm}число, месяц, год\\
Задание получил~\makebox[4cm]{\hrulefill}~~~~~\makebox[5cm]{\hrulefill}\\
\hspace{4.2cm} подпись, дата \hspace{1.4cm} инициалы, фамилия
\end{flushleft}

\end{document}
