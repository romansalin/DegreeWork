\documentclass[a4paper,14pt]{extarticle}

\usepackage{cmap} % improved search for russian words in pdf

% Nice cyrillic fonts
\usepackage{pscyr}
\renewcommand{\rmdefault}{ftm} % Times New Roman
%\renewcommand{\sfdefault}{ftx}
%\renewcommand{\ttdefault}{cmttp} % not good ttf

% Links
\usepackage{hyperref}
\hypersetup{
        unicode=true,
        colorlinks,
        citecolor=black,
        filecolor=black,
        linkcolor=black,
        urlcolor=blue
}
\usepackage[all]{hypcap} % fix links to floats

\usepackage{mathtext}
\usepackage[mathscr]{eucal}

\usepackage[T2A]{fontenc} % Russian support
\DeclareSymbolFont{T2Aletters}{T2A}{cmr}{m}{it}
\usepackage[utf8]{inputenc} % UTF8
\usepackage[english,russian]{babel}

% Mathematical AMS packages
\usepackage{amsmath, amsfonts, amsthm, amssymb, amscd}

% Provides support for setting the spacing between lines in a document. Package
% options include singlespacing, onehalfspacing, and doublespacing.
\usepackage{setspace}
\onehalfspacing % one half spacing

\usepackage{indentfirst} % indent
\setlength{\parindent}{1.25cm}

% Alternative geometry
\usepackage[top=2cm, bottom=2cm, left=2.5cm, right=1.5cm, bindingoffset=0cm,
 			headheight=0cm, footskip=1cm, headsep=0cm]{geometry}

% Nice citations [1,2,3,4] -> [1-4]
\usepackage[numbers,sort&compress]{natbib}

\usepackage{soul} % hyphenation for letterspacing, underlining and more

\sloppy % makes TeX less fussy about line breaking

% Support for the upright and bold greek letters
\usepackage{bm}
\usepackage[Symbolsmallscale]{upgreek}
\makeatletter
        \newcommand{\bfgreek}[1]{\bm{\@nameuse{up#1}}}
\makeatother

\usepackage{graphicx} % for graphics
\graphicspath{{images/}} % images path

\usepackage{tikz} % for drawing
\usetikzlibrary{shapes,arrows}

\usepackage{titlesec} % select alternative section titles
\usepackage{titletoc} % alternative headings for toc/lof/lot

% Keeps floats `in their place', preventing them from floating past a
% "\FloatBarrier" command into another section.  The floats should not move
% past every "\section".
\usepackage[section]{placeins}

\usepackage{longtable} % long table support
\usepackage{multirow,makecell,array}	 % advanced table style
\usepackage{tabularx}

\usepackage{float}

% Useful for individually placing figures on a separate page with
% \afterpage{\clearpage \begin{figure}[p] ... }
\usepackage{afterpage}


% --------------------------------------------------------------------------%
% Numeration of pages
% --------------------------------------------------------------------------%
\pagestyle{headings}
\makeatletter
\renewcommand{\@oddhead}{}
\renewcommand{\@oddfoot}{\hfil \thepage}
\setcounter{tocdepth}{2}
% --------------------------------------------------------------------------%


% --------------------------------------------------------------------------%
% Sections, subsections
% --------------------------------------------------------------------------%
% Numbering
\renewcommand{\thesection}{\arabic{section}}
\renewcommand{\thesubsection}{\arabic{section}.\arabic{subsection}}
\renewcommand{\thesubsubsection}
        {\arabic{section}.\arabic{subsection}.\arabic{subsubsection}}

\newcommand{\sectionbreak}{\clearpage}

% Contents, intro, conclusion
\newcommand{\structformat}
{
   \titleformat{\section}[block]
   {\centering\bfseries}
   {\thesection }{}{}
}

% Sections, subsections
\newcommand{\secformat}
{
    \titleformat{\section}[block]
    {\hspace{1.25cm}\raggedright\bfseries}
    {\thesection}{1ex}{}
}

\newcommand{\starsection}[1]{
    \structformat
    \section*{#1}
    \addcontentsline{toc}{section}{#1}
    \setcounter{section}{0}
    \secformat
}

\newcommand{\intro}{\starsection{ВВЕДЕНИЕ}}
\newcommand{\conclusion}{\starsection{ЗАКЛЮЧЕНИЕ}}

\titleformat{\subsection}[block]{\hspace{1.25cm}\normalfont\raggedright\bfseries}
		{\thesubsection}{1ex}{}
\titleformat{\subsubsection}[block]{\hspace{1.25cm}\normalfont\raggedright}
		{\thesubsubsection}{1ex}{}

\titlespacing*{\section}{0pt}{3ex plus 1ex minus .2ex}{3ex plus.2ex}
\titlespacing*{\subsection}{0pt}{2ex plus 1ex minus .2ex}{.3ex plus.2ex}
\titlespacing*{\subsubsection}{0pt}{2ex plus 1ex minus .2ex}{.3ex plus.2ex}
% --------------------------------------------------------------------------%


% --------------------------------------------------------------------------%
% Table and figure captions
% --------------------------------------------------------------------------%
\usepackage{caption}
\def\CaptionName#1{\gdef\@captionname{#1}}
\newlength\tmp %10cm
\setlength{\tmp}{1ex}
\setlength{\belowcaptionskip}{1ex}
\setlength{\abovecaptionskip}{1ex}

\captionsetup[table]{name=Таблица, labelsep=endash, format=plain, justification=RaggedRight,
			singlelinecheck=false, font={small}, position=top}
\captionsetup[figure]{name=Рисунок, labelsep=endash, justification=centering,
			font={small}, skip=\abovecaptionskip, position=below}
% --------------------------------------------------------------------------%


% --------------------------------------------------------------------------%
% Table and figure numbering by sections
% --------------------------------------------------------------------------%
\renewcommand{\theequation}{\arabic{section}.\arabic{equation}}
\renewcommand{\thefigure}{\arabic{section}.\arabic{figure}}
\renewcommand{\thetable}{\arabic{section}.\arabic{table}}

\makeatletter
\@addtoreset{equation}{section} % Equation counter
\@addtoreset{figure}{section} % Figure counter
\@addtoreset{table}{section} % Table counter
\makeatother
% --------------------------------------------------------------------------%


% --------------------------------------------------------------------------%
% Theorem, proof, definition, lemma, proposition, corollary
% --------------------------------------------------------------------------%
\newtheoremstyle{note}  % name
     {3mm}              % Space above
     {3mm}              % Space below
     {}                 % Body font
     {\parindent}       % Indent amount (empty = no indent, \parindent = para indent)
     {\bfseries}        % Thm head font
     {.}                % Punctuation after thm head
     { }                % Space after thm head: " " = normal interword space; \newline = linebreak
     {}                 % Thm head spec (can be left empty, meaning 'normal')

\theoremstyle{note}

\newtheorem{definition}{Определение}
\newtheorem{theorem}{Теорема}
\newtheorem{lemma}{Лемма}
\newtheorem{proposition}{Предложение}
\newtheorem{corollary}{Следствие}

\renewcommand{\proof}{\textbf{Доказательство.}\ignorespaces{\pushQED{\qed}}}
% --------------------------------------------------------------------------%


% --------------------------------------------------------------------------%
% Enumerations
% --------------------------------------------------------------------------%
\makeatletter
\renewcommand\theenumi  {\@arabic\c@enumi}
\renewcommand\theenumii {\@asbuk\c@enumii}
\renewcommand\theenumiii{\@roman\c@enumiii}
\renewcommand\theenumiv {\@Asbuk\c@enumiv}
\newcommand\atheenumi{\@asbuk\c@enumi}
\newcommand\atheenumii{\@asbuk\c@enumii}
\renewcommand\labelenumi  {\theenumi.}
\renewcommand\labelenumii {\theenumii.}
\renewcommand\labelenumiii{\theenumiii.}
\renewcommand\labelenumiv {\theenumiv.}
\renewcommand\p@enumii  {\theenumi}
\renewcommand\p@enumiii {\theenumi.\theenumii}
\renewcommand\p@enumiv  {\p@enumiii.\theenumiii}
\renewcommand\labelitemi  {\normalfont\bfseries\textemdash}
\renewcommand\labelitemii {\normalfont\bfseries\textendash}
\renewcommand\labelitemiii{\textperiodcentered}
\renewcommand\labelitemiv {\textasteriskcentered}

\renewcommand{\@listI}{%
\leftmargin=52pt
\rightmargin=0pt
\labelsep=7pt
\labelwidth=20pt
\itemindent=0pt
\listparindent=0pt
\topsep=4pt plus 2pt minus 4pt
\partopsep=0pt plus 1pt minus 1pt
\parsep=0pt plus 1pt
\itemsep=\parsep}
\makeatother

% Compressed lists: compactitem etc.
\usepackage{paralist}

\usepackage{enumitem}
\setlist[itemize]{fullwidth, listparindent=\parindent}
\setlist[enumerate]{fullwidth, itemindent=\parindent, listparindent=\parindent}
% --------------------------------------------------------------------------%

\begin{document}

\intro

В данной выпускной квалификационной работе рассматривается сеть массового обслуживания с зависимой от состояния сети маршрутизацией как модель гибкой производственной системы. В сети используется вероятностная маршрутизация в кратчайшую очередь (PSQ--маршрутизация). При данной маршрутизации маршрутные вероятности больше для систем с б\'{о}льшим числом свободных приборов. В этом случае стационарное распределение вероятностей состояний сети имеет мультипликативную форму, для которого имеется эффективный алгоритм вычисления. Важно отметить, что в рассматриваемой модели допускаются как ограниченные, так и неограниченные очереди в системах.

Целью выпускной квалификационной работы является исследование и анализ производственных систем с маршрутизацией. Задачами являются: разработка алгоритма метода анализа данных производственных систем, программная реализация алгоритма и проведение численных экспериментов с разработанной программой.


%%%%%%%%%%%%%%%%%%%%%%%%%%%%%%%%%%%%%%%%%%%%%%%%%%%%%%%%%%%%%%%%%%%%%%%%%%%%%%%


\section{Гибкие производственные системы с маршрутизацией, зависящей от состояния}
\label{sec:FMS}

\subsection{Описание модели}
\label{subsec:FMS_specification}

Рассмотрим гибкую производственную систему (ГПС), основные компоненты которой следующие.

\begin{enumerate}
\item Множество рабочих станций (систем) $C_i$ с номерами из множества $I \equiv \{ i ~|~ i=1,...,L \}$. Каждая рабочая станция $C_i$ производит обработку деталей и может выполнять один или несколько типов производственных операций $t$ (например, сверление, фрезерование, нарезка, растачивание и т.д.). На станции $C_i$ есть $\kappa_i$ параллельно работающих машин (приборов). Максимальное число деталей, которое допускается в любой момент времени (включая как обрабатывающиеся детали, так и детали, ожидающие в локальном хранилище), ограничены $s_i$ (емкость рабочей станции), где $s_i \geqslant \kappa_i, ~ i=1,...,L$. Определим вектор числа приборов в рабочих станциях $\kappa = (\kappa_i)$ и вектор емкостей рабочих станций $s = (s_i)$, $i=1,...,L$.

\item Система транспортировки материалов (MHS), обозначаемая как станция $C_0$, которая состоит из $\kappa_0$ транспортеров (тележки, конвейеры и т.д.), которые осуществляют транспортировку деталей между рабочими станциями, и центрального хранилища. Помимо транспортировки на этой станции выполняется погрузка/разгрузка деталей (на и из палет), закрепление и т.д. То есть вся подготовительная работа, которая делается в период между последовательными посещениями детали двух различных станций).

\item Палеты, на которых перемещаются детали. Для каждой детали выделяется одна палета. Общее количество палет, доступных в ГПС, постоянно и равно $N$. Для удобства рассуждений будем считать, что $N \leqslant \sum\limits_{i=1}^L s_i$. В противном случае, в центральном хранилище всегда будут существовать по крайней мере $N - \sum\limits_{i=1}^L s_i$ деталей (эта ситуация может быть учтена в модели исключением соответствующих точек из пространства состояний).
\end{enumerate}

Схема гибкой производственной системы представлена на рисунке \ref{fig:fms}.

Функционирование гибкой производственной системы происходит следующим образом.

Система должна обработать $t=1,2,...,T$ типов деталей. Общее число деталей в системе в любой момент времени постоянно и равно $N$. Другими словами, все доступные палеты все время заняты. Всякий раз, когда обработанная деталь покидает систему, другая деталь того же типа сразу же поступает в систему. Это неявно предполагает, что имеется непрерывное снабжение системы деталями для их обработки. Далее будет рассматриваться только долгосрочное поведение гибкой производственной системы или стационарный режим функционирования. Число палет $N_t$, выделяемых для деталей типа $t$, постоянно и $\sum\limits_t N_t = N$. Определим вектор начального числа деталей $\mathbf{N}=(N_t)$, $t=1,...,T$.

Рабочие станции или машины могут быть свободны (простаивать), но они никогда не блокируются. На практике это, как правило, обеспечивается за счет \textit{возвратного конвейера}, который постоянно забирает из рабочих станций детали, завершившие обработку, и доставляет их обратно в центральное хранилище (предполагается, что в центральном хранилище достаточно мест для размещения всех деталей в случае необходимости, т.е. $s_0=N$). Задержка деталей на возвратном конвейере не учитывается. В этом смысле емкость хранилища возвратного конвейера является частью емкости центрального хранилища.

Деталь каждого типа $t$ нуждается во множестве операций, которые будут выполняться на наборе рабочих станций с номерами $I_t \subseteq I$. Длительность обработки детали типа $t$ на станции $C_i$, $i \in I_t$, имеет экспоненциальное распределение с параметром $\mu_{it}$. Пусть $s_{it}$~--- емкость места хранения, выделенного для деталей типа $t$ на станции $C_i$. Пусть $s_{it}=0$, если $i \notin I_t$, тогда $\sum\limits_t s_{it} = s_i$.

Длительность обработки детали на станции $C_0$ включает в себя время перехода детали на следующую станцию, а также время для погрузки/разгрузки и повторной установки в случае необходимости. Распределение длительности обработки деталей типа $t$ также предполагается экспоненциальным с параметром $\mu_{0t}$.

Дисциплина обработки на всех станциях $C_i,~i=0,...,L$,~--- \textit{RANDOM} (то есть детали выбираются для обработки случайным образом). Для определенности предполагается, что для каждой станции $C_i$ среднее число приборов (механизмов/транспортеров), обрабатывающих детали типа $t$, пропорционально $n_{it}$~--- количеству деталей типа $t$ на станции $C_i$. Определим вектор размерности $Т$
\begin{equation*}
 \overline{\eta}_i = (n_{i1},...,n_{iT}), \quad \text{где}~ n_{it}=0, ~\text{если}~ i \notin I_t, ~\text{и пусть}~ n_i = \sum_{t=1}^{T} n_{it}.
\end{equation*}
Вектор $\overline{\eta}_i$ определяет состояние станции $C_i$ как вектор числа деталей различных типов, находящихся на этой станции.
Суммарная интенсивность обработки деталей типа $t$ на станции $C_i$ может быть выражена в виде
\begin{equation*}
 \mu_{it} n_{it} \nu_{i}(n_i), \quad \text{где} ~\nu_i(n_i) = \frac{\min(n_i, \kappa_i)}{n_i}.
\end{equation*}

На станции $C_0$ решения, связанные с маршрутизацией, принимаются всякий раз, когда транспортер становится доступным. Согласно описанной выше дисциплине обслуживания \textit{RANDOM}, если деталь типа $t$ должна быть доставлена следующей, то для доставки выбирается некоторая станция с номером $i \in I_t$, которая имеет наибольшее число свободных мест, т.е. $s_{it} - n_{it} \geqslant s_{kt} - n_{kt}$ для всех $k \in I_t$ и $k \neq i$. Одна из деталей типа $t$, ожидающая в центральном хранилище обработки на станции $C_i$ (эту деталь будем называть <<очередной>> деталью), затем доставляется на станцию $C_i$. Эта схема маршрутизации называется детерминированной маршрутизацией в кратчайшую очередь или DSQ--маршрутизацией.

Пусть $\Theta = (\theta_{it,jt})$~--- маршрутная матрица, $t=1,...,T$, $i,j=0,...,L$, где $\theta_{it,jt}$~--- вероятность того, что деталь типа $t$ после обработки на станции $C_i$ поступает на станцию $C_j$. Заметим, что допускаются переходы только из системы транспортировки на рабочие станции и из рабочих станций в систему транспортировки, т.е. $\theta_{it,jt}=0$ при $i,j>0$. Также заметим, что деталь типа $t$ может перемещаться только на рабочую станцию из множества $I_t \subseteq I$, то есть $\theta_{it,jt'}=0$ при $t \neq t'$.

Данная гибкая производственная система с введенными выше предположениями описывается неоднородной замкнутой экспоненциальной сетью массового обслуживания $\Gamma=\left<L,T,\mathbf{N},N,M,\Theta,\kappa,\mu,\textit{RANDOM}\right>$. В дальнейшем термины <<гибкая производственная система>>, <<рабочая станция>>, <<обрабатывающий прибор>>, <<детали>>, <<обработка деталей>>, использующиеся для описания гибких производственных систем, будем отождествлять соответственно с терминами <<сеть массового обслуживания>>, <<система>>, <<обслуживающий прибор>>, <<требования>>, <<обслуживание требований>>.

\subsection{Решение уравнения равновесия}
\label{subsec:solution}

Определим состояние сети $\Gamma$, соответствующей рассмотренной выше гибкой производственной системе, как $\overline{\eta} = (\overline{\eta}_0, \overline{\eta}_1,...,\overline{\eta}_L)$, где $\overline{\eta}_i = (n_{i1},...,n_{iT})$, $i=0,1,...,L$, обозначает вектор числа требований в системе $C_i$. Из предположений в разделе~\ref{subsec:FMS_specification} нетрудно видеть, что $\{ \overline{\eta}(\tau) \}$ определяет процесс Маркова со следующим конечным пространством состояний:
\begin{equation}
 S = \left\lbrace \overline{\eta} \in Z_+^{(L+1)T} ~|~ n_{it} \leqslant s_{it} ~ (i \in I), ~
 \sum_{i \in I_t^+} n_{it} = N_{t}, ~ t=1,...,T \right\rbrace ,
 \label{eq:2.1}
\end{equation}
где $Z_+$ обозначает множество неотрицательных чисел и $I_t^+ = \{ 0 \} \cup I_t$.

Сформулируем PSQ--маршрутизацию следующим образом. Вероятности перехода требований класса $t$ из системы $C_0$ в систему $C_i$, $i \in I_t$, зависят от $n_{0t}$ и от $n_{it}$~--- числа требований класса $t$ в двух системах, и принимают форму
\begin{equation}
 \theta_{0t,it} = \frac{r_{it}(n_{it})} {r_{0t}(n_{0t})},
\label{eq:2.2}
\end{equation}
где $r_{it}(\cdot)$ и $r_{0t}(\cdot)$~--- две линейные функции:
\begin{equation*}
 r_{it}(n_{it}) = s_{it} - n_{it} ~ \text{и} ~ r_{0t}(n_{0t}) = \sum_{C_i \in I_t} s_{it} + n_{0t} - N_t .
\end{equation*}
Заметим, что $r_{it}(\cdot)$ показывают число свободных мест в системе $C_i$ для требований класса $t$, а $r_{0t}(n_{0t}) = \sum\limits_{C_i \in I_t} r_{it}(n_{it})$ такие, что маршрутные вероятности в сумме по всем $i \in I_t$ равны единице. Несложно заметить следующие особенности этой схемы маршрутизации:
\begin{itemize}
\item маршрутные вероятности выше для систем с б\'{о}льшим числом свободных приборов;
\item требования класса $t$ никогда (т.е. с вероятностью ноль) не направляются в систему, в которой все места в очереди для ожидания требованиями этого класса заняты (т.е. когда $n_{it} = s_{it}$).
\end{itemize}

Стационарное решение для сети $\Gamma$ как модели гибкой производственной системы можно теперь обобщить следующим образом.

\begin{theorem}
 Марковский процесс $\overline{\eta}(\tau)$, определенный в пространстве состояний $S$ и управляемый PSQ--маршрутизацией, как определено в~(\ref{eq:2.2}), является обратимым относительно времени и имеет следующую мультипликативную форму стационарного распределения вероятностей:
 \begin{equation}
  \pi(\overline{\eta}) = G^{-1} \prod_{i=0}^L \left[ \prod_{j=1}^{n_i} \nu_i^{-1} (j) \right]
  \left[ \prod_{t=1}^T \prod_{j=1}^{n_{it}} \frac{r_{it} (j - 1 + \delta_{i0})}{j\mu_{it}} \right], \quad \overline{\eta} \in S ,
  \label{eq:2.4}
 \end{equation}
где $\delta_{i0}=1$, если $i=0$, иначе $\delta_{i0}=0$, и $G$~--- нормализующая константа.
\end{theorem}


%%%%%%%%%%%%%%%%%%%%%%%%%%%%%%%%%%%%%%%%%%%%%%%%%%%%%%%%%%%%%%%%%%%%%%%%%%%%%%%


\section{Алгоритм метода анализа производственных систем с~маршрутизацией, зависящей от состояния}
\label{sec:algorithm}

\subsection{Описание алгоритма}
\label{subsec:algorithm_description}

Рассмотрим неоднородную замкнутую экспоненциальную сеть массового обслуживания $\Gamma'~=~\langle L, T, N, M, \Theta, \kappa, \mu, FCFS \rangle$ с $L$ системами $C_i$, $i=1,...,L$, (включая систему транспортировки материалов, которая теперь обозначается как любая $C_i$) и $N$ требованиями (палетами в ГПС).

Не теряя общности, предположим, что требуется получить предельное маргинальное (частное) распределение вероятностей $\pi_L(k,N)~(k=0,1,...,s_L)$ для системы $C_L$ сети $\Gamma'$ с $L$ системами и $N$ требованиями.

Определим векторы размерности $T$: $\mathbf{N}=(N_1,...,N_T)$, $\mathbf{n}=(n_1,...,n_T)$, $\mathbf{e_t}=(0,...,1,...,0)$ ($t$--ый элемент равен $1$, а все остальные нули) и $\mathbf{0}=(0,...,0)$. Пусть $G(m,\mathbf{N})$ будет нормализующей константой, где $\mathbf{N}$~--- вектор начального числа требований в сети. Кроме того, определим
\begin{equation*}
R_t(m,\mathbf{N}) = \frac{G(m,\mathbf{N}-\mathbf{e_t})}{G(m,\mathbf{N})},~~t=1,...,T.
\end{equation*}
Обозначим через $\pi_m(\mathbf{n},\mathbf{N})$ предельную вероятность того, что общее число требований в системе $m$ будет равно $\mathbf{n}$, когда общее число требований всей сети равно $\mathbf{N}$. Тогда справедливо следующее следствие.

\begin{corollary}
 Если $\mathbf{n} > \mathbf{0}$ (т.е. все элементы вектора $\mathbf{n}$ положительны), то для всех $t=1,...,T$
 \begin{equation}
  \pi_m(\mathbf{n},\mathbf{N}) = \pi_m(\mathbf{n}-\mathbf{e_t},\mathbf{N}-\mathbf{e_t})
  f_{mt}(\mathbf{n}) R_t(m,\mathbf{N}) ,
  \label{eq:10}
 \end{equation}
 иначе для всех $t=1,...,T$
  \begin{equation}
  \pi_m(\mathbf{n},\mathbf{N}) = \pi_m(\mathbf{n},\mathbf{N}-\mathbf{e_t}) R_t^{-1}(m-1,\mathbf{N}-\mathbf{n}) R_t(m,\mathbf{N}) .
  \label{eq:11}
 \end{equation}
\end{corollary}

Разработанный ранее алгоритм применим и здесь со следующими изменениями.
\begin{enumerate}
\item Необходимы два двухмерных массива для $\pi(\mathbf{n})$ и $R_t$, $t=1,...,T$.
\item В этом случае имеется $T$ нормализующих множителей $R_t$, $t=1,...,T$, которые в дополнение к условию нормировки значений $\pi_m$ удовлетворяют следующим уравнениям:
\begin{equation}
 R_t(m,\mathbf{N}) = R_t(m-1,\mathbf{N}) \frac{\pi_m(\mathbf{0},\mathbf{N})}
 {\pi_m(\mathbf{0},\mathbf{N}-\mathbf{e_t})}, \quad t=1,...,T .
 \label{eq:12}
\end{equation}
Эти уравнения следуют непосредственно из~(\ref{eq:11}) при предположении, что $\mathbf{n}=\mathbf{0}$.
\end{enumerate}

Алгоритм для неоднородной сети приведен в следующем разделе.

\subsection{Структурная схема алгоритма}
\label{subsec:flowchart}

Алгоритм метода анализа сети массового обслуживания $\Gamma$ с маршрутизацией, зависящей от состояния, имеет блочную структуру, представленную на рисунке \ref{img:4.1}.

\begin{figure}[H]
\centering
\tikzstyle{arrow} = [draw, ->, >=angle 60]
\tikzstyle{block} = [rectangle, draw, text width=16em, text centered]
\tikzstyle{inout} = [trapezium, draw, text width=10em, text centered, trapezium left angle=70,trapezium right angle=-70]
\tikzstyle{for} = [shape=chamfered rectangle, chamfered rectangle xsep=2cm, draw]
\begin{tikzpicture}[node distance = 2cm, auto]
\node [rounded rectangle, draw] (begin) {Начало};
\node [inout, below of=begin] (init) {Блок 1. Ввод исходных данных};
\node [for, below of=init] (for) {$i=1,2,...,L$};
\node [block, below of=for] (renum) {Блок 2. Перестановка СМО $C_i$ и $C_L$};
\node [block, below of=renum, node distance = 2.5cm] (pi) {Блок 3. Вычисление стационарного распределения вероятностей СМО $C_i$};
\node [block, below of=pi, node distance = 2.5cm] (renum_back) {Блок 4. Обратная перестановка СМО $C_L$ и $C_i$};
\node [block, below of=renum_back, node distance = 4.5cm] (parameters) {Блок 5. Вычисление стационарных характеристик СеМО};
\node [inout, below of=parameters, node distance = 2.5cm] (output) {Блок 6. Вывод результатов};
\node [rounded rectangle, draw, below of=output] (end) {Конец};
\path [arrow] (begin) -- (init);
\path [arrow] (init) -- (for);
\path [arrow] (for) -- (renum);
\path [arrow] (renum) -- (pi);
\path [arrow] (pi) -- (renum_back);
\path [arrow] (renum_back) -- ++(0,-2) -- ++(-6,0) |- (for);
\path [arrow] (for) -- ++(6,0) -- ++(0,-10) -| (parameters);
\path [arrow] (parameters) -- (output);
\path [arrow] (output) -- (end);
\end{tikzpicture}
\caption{Блок-схема алгоритма метода анализа сетей массового обслуживания с маршрутизацией, зависящей от состояния}
\label{img:4.1}
\end{figure}

\medskip
\textbf{Блок 1. Ввод исходных данных}

На начальном этапе работы алгоритма вводятся параметры сети массового обслуживания $\Gamma$:\\
$L$~--- число СМО в СеМО;\\
$\mathbf{N}=(N_t)$~--- вектор начального числа требований в СеМО, $t=1,...,T$;\\
$\kappa=(\kappa_i)$~--- вектор числа приборов в системах обслуживания СеМО, $i=0,...,L$;\\
$s=(s_{it})$~--- матрица емкостей систем в СеМО, $i=0,...,L,~t=1,...,T$;\\
$\mu=(\mu_{it})$~--- матрица интенсивностей обслуживания требований системами СеМО, $i=0,...,L,~t=1,...,T$.

\medskip
\textbf{Блок 2. Перестановка СМО $\boldsymbol{C_i}$ и $\boldsymbol{C_L}$}

Во втором блоке для вычисления стационарного распределения вероятностей состояний системы $C_i,~i=1,...,L,$ происходит перестановка системы $C_i$ с последней системой $C_L$. Это осуществляется путем перестановки элементов $\kappa_i$ и $\kappa_L$ в векторе $\kappa$, $s_{it}$ и $s_{iL}$ в матрице $s$, $\mu_{it}$ и $\mu_{iL}$ в матрице $\mu$, $i=1,...,L,~t=1,...,T$. \\
\textit{Входные данные:} $\kappa=(0,...,\kappa_i,...,\kappa_L)$, $s=\left( \begin{matrix}
s_{1t}\\
\cdots\\
s_{it}\\
\cdots\\
s_{Lt}
\end{matrix} \right )$, $\mu=\left( \begin{matrix}
\mu_{1t}\\
\cdots\\
\mu_{it}\\
\cdots\\
\mu_{Lt}
\end{matrix} \right )$, $t~=~1,...,T$. \\
\textit{Выходные данные:} $\kappa'=(0,...,\kappa_L,...,\kappa_i)$, $s'=\left( \begin{matrix}
s_{1t}\\
\cdots\\
s_{Lt}\\
\cdots\\
s_{it}
\end{matrix} \right )$, $\mu'=\left( \begin{matrix}
\mu_{1t}\\
\cdots\\
\mu_{Lt}\\
\cdots\\
\mu_{it}
\end{matrix} \right )$, $t=1,...,T$.

\medskip
\textbf{Блок 3. Вычисление стационарного распределения вероятностей состояний СМО $\boldsymbol{C_i}$}

Алгоритм для вычисления стационарного распределения вероятностей состояний для системы $C_i$ следующий:\\
\emph{Входные данные:} $L$, $T$, $\mathbf{N}=(N_t)$, $\kappa=(\kappa_i)$, $s=(s_{it})$, $\mu=(\mu_{it})$, $i=0,...,L,~t=1,...,T$.
\emph{Выходные данные:} $\pi_i(\mathbf{n},\mathbf{N})$, $i=1,...,L$.

\medskip
\textbf{Блок 4. Обратная перестановка СМО $\boldsymbol{C_L}$ и $\boldsymbol{C_i}$}

Происходит обратная перестановка системы $C_ L$ с системой $C_i$. Таким образом, получаем исходный вектор $\kappa$ и матрицы $s$, $\mu$. \\
\textit{Входные данные:} $\kappa'=(0,...,\kappa_L,...,\kappa_i)$, $s'=\left( \begin{matrix}
s_{1t}\\
\cdots\\
s_{Lt}\\
\cdots\\
s_{it}
\end{matrix} \right )$, $\mu'=\left( \begin{matrix}
\mu_{1t}\\
\cdots\\
\mu_{Lt}\\
\cdots\\
\mu_{it}
\end{matrix} \right )$, $t=1,...,T$. \\
\textit{Выходные данные:} $\kappa=(0,...,\kappa_i,...,\kappa_L)$, $s=\left( \begin{matrix}
s_{1t}\\
\cdots\\
s_{it}\\
\cdots\\
s_{Lt}
\end{matrix} \right )$, $\mu=\left( \begin{matrix}
\mu_{1t}\\
\cdots\\
\mu_{it}\\
\cdots\\
\mu_{Lt}
\end{matrix} \right )$, $t~=~1,...,T$.

\medskip
\textbf{Блок 5. Вычисление стационарных характеристик СеМО}
\emph{Входные данные:} $L$, $T$, $\mathbf{N}=(N_t)$, $\pi_m(\mathbf{n},\mathbf{N})$, $\kappa=(\kappa_i)$, $s=(s_{it})$, $\mu=(\mu_{it})$, $i=0,...,L,~m=1,...,L,~t=1,...,T$.\\
\emph{Выходные данные:} $\overline{n}_{it}$, $\lambda_{it}$, $\psi_{it}$, $i=0,...,L,~t=1,...,T$.

На данном этапе происходит вычисление следующих стационарных характеристик СеМО:
\begin{itemize}
\item м. о. числа $t$-требований в СМО;
\item м. о. числа занятых $t$-требованиями приборов в СМО;
\item интенсивность входящего потока $t$-требований в СМО;
\item коэффициенты использования обслуживающих приборов СМО $t$~-~требованиями.
\end{itemize}
Эти характеристики вычисляются по формулам (\ref{eq:distribution_n}) -- (\ref{eq:distribution_psi0}).\\
\begin{equation}
\overline{n}_{it} = \sum\limits_{k=0}^{s_{it}} k \pi_i(\mathbf{n},\mathbf{N}),
 \label{eq:distribution_n}
\end{equation}
\begin{equation}
\overline{h}_{it} = \left\{
 \begin{array}{l}
 \sum\limits_{k=0}^{\kappa_i} k \pi_i(\mathbf{n},\mathbf{N}) + \kappa_i  \sum\limits_{k=\kappa_i + 1}^{s_{it}} \pi_i(\mathbf{n},\mathbf{N}), \quad \kappa_i < s_{it}, \\
 \sum\limits_{k=0}^{s_{it}} k \pi_i(\mathbf{n},\mathbf{N}), \quad \kappa_i \ge s_{it},
 \end{array}
\right.
 \label{eq:distribution_h}
\end{equation}
\begin{equation}
\lambda_{it} = \mu_{it} \overline{h}_{it},
 \label{eq:distribution_lambda}
\end{equation}
\begin{equation}
\psi_{it} = \frac{\lambda_{it}}{\min(\kappa_{i}, s_{it}) \mu_{it}},
 \label{eq:distribution_psi}
\end{equation}
где $i=1,...,L,~t=1,...,T$.
\begin{equation}
\overline{n}_{0t} = N_t - \sum\limits_{i=1}^L \overline{n}_{it},
 \label{eq:distribution_n0}
\end{equation}
\begin{equation}
\lambda_{0t} = \sum\limits_{i=1}^L \lambda_{it},
 \label{eq:distribution_lambda0}
\end{equation}
\begin{equation}
\psi_{0t} = \frac{\lambda_{0t}}{\kappa_0 \mu_{0t}},
 \label{eq:distribution_psi0}
\end{equation}
где $t=1,...,T$.

\medskip
\textbf{Блок 6. Вывод результатов}

В данном блоке происходит вывод (на экран или в файл) стационарного распределения и стационарных характеристик, полученных в блоке 4 и 5.


%%%%%%%%%%%%%%%%%%%%%%%%%%%%%%%%%%%%%%%%%%%%%%%%%%%%%%%%%%%%%%%%%%%%%%%%%%%%%%%


\section{Описание и назначение программы}
\label{sec:program_description_and_purpose}

Программа, предназначенная для  анализа производственных систем с маршрутизацией, зависящей от состояния, была реализована на языке программирования Java.

Программа позволяет вычислить стационарное распределение и основные характеристики производственных систем с маршрутизацией, зависящей от состояния. Вычисления могут производится как для однородной, так и для неоднородной сети массового обслуживания. В случае необходимости произвести вычисления для однородной сети, в окне программы или во входном файле необходимо положить $T=1$.

Разработанная программа имеет графический интерфейс. Входные данные считываются с формы, проверяются на корректность, и в соответствии с проверкой либо производится анализ, либо выдается сообщение об ошибке. Для рассматриваемой сети входными данными являются: число систем, число классов требований, вектор числа требований определенного класса, вектор числа обслуживающих приборов, емкости систем и интенсивности обслуживания. Выходными данными являются стационарное распределение и основные стационарные характеристики систем, а именно: м. о. числа требований, интенсивности потока требований и коэффициенты использования обслуживающих приборов.

При запуске программы появляется окно, изображенное на рисунке \ref{fig:main}. Для удобства существует возможность открыть файл с заданными в нем входными данными.

Для анализа при заданных входных данных служит кнопка <<Получить результаты>>. При ее нажатии открывается окно с подсчитанными в ходе анализа основными характеристиками СМО и стационарным распределением (см. рисунок \ref{fig:results}). В случае введения некорректных начальных данных, на экран будет выведено сообщение об ошибке.

В окне результатов анализа пользователю предлагается на выбор два действия: закрыть окно или сохранить результаты в файл. Характеристики будут сохранены в указанный файл.


%%%%%%%%%%%%%%%%%%%%%%%%%%%%%%%%%%%%%%%%%%%%%%%%%%%%%%%%%%%%%%%%%%%%%%%%%%%%%%%


\section{Аспекты практического применения}
\label{sec:practical_application}

Было проведено несколько серий экспериментов с использованием разработанной программы анализа производственных систем с маршрутизацией, зависящей от состояния.

\textbf{Эксперимент 3}

Сравним PSQ--маршрутизацию с DSQ--маршрутизацией. Рассмотрим производственную систему с $18$ машинами, которые сгруппированы по $9$ рабочим станциям. Число приборов, интенсивность обработки детали одним прибором и емкость локального хранилища на каждой станции соответственно равны: \\
$\kappa_1=\kappa_2=\kappa_3=1$; $\kappa_4=\kappa_5=\kappa_6=2$; $\kappa_7=\kappa_8=\kappa_9=3$; \\
$\mu_1=\mu_2=\mu_3=2$; $\mu_4=\mu_5=\mu_6=1,5$; $\mu_7=\mu_8=\mu_9=1$; \\
$s_1=s_2=s_3=4$; $s_4=s_5=s_6=6$; $s_7=s_8=s_9=7$. \\
Число транспортеров $\kappa_0=9$, каждый из которых имеет интенсивность обработки $\mu_0=3$. В данной системе есть $N=50$ палет, т.е. общее число деталей (одного типа) в любой момент времени равно $50$. Мы сравним производительность производственной системы, имеющей PSQ--маршрутизацию, с производственной системой, имеющей DSQ--маршрутизацию, где вероятность перехода от станции $C_0$ к станциям $C_1$, $C_2$ и $C_3$ равна $2/24$, а к другим станциям~--- $3/24$. 

Математическое ожидание числа деталей ($\overline{n}_i$), пропускная способность ($\lambda_i$) и коэффициенты использования приборов ($\psi_i$) для каждой станции приведены в таблице~\ref{tab:result3} (в скобках указаны результаты для DSQ--маршрутизации). PSQ--маршрутизация имеет очевидные преимущества с точки зрения увеличения пропускной способности, а также коэффициентов использования приборов и очередей производственной системы \cite{yao1}.

{\renewcommand{\arraystretch}{1.5}%
\begin{table}[H]
\caption{} \label{tab:result3}
\begin{tabular}{|c|c|c|c|c|}
\hline
$C_i$ & $C_0$ & $C_{1, 2, 3}$ & $C_{4, 5, 6}$ & $C_{7, 8, 9}$ \cr
\hline
$\overline{n}_i$  &  11,070 (18,937)  &  3,003 (2,230)  &  4,492 (3,638)  &  5,482 (4,486) \cr
\hline
$\lambda_i$  &  23,595 (21,168)  &  1,954 (1,687)  &  2,950 (2,672)  &  2,961 (2,697) \cr
\hline
$\psi_i$  &  0,874 (0,784)  &  0,977 (0,844)  &  0,983 (0,891)  &  0,987 (0,899) \cr
\hline
\end{tabular}
\end{table}}

\textbf{Эксперимент 5}

Возьмем гибкую производственную систему из примера 3 и посмотрим, как будут изменяться характеристики рабочих станций $C_i$, $i=1,2$, с изменением числа приборов, емкостей рабочих станций и интенсивностей обработки деталей (эти данные предполагаются одинаковыми для обеих рабочих станций). Результаты представлены на рисунках \ref{fig:graph1}--\ref{fig:graph3}.


%%%%%%%%%%%%%%%%%%%%%%%%%%%%%%%%%%%%%%%%%%%%%%%%%%%%%%%%%%%%%%%%%%%%%%%%%%%%%%%


\conclusion
Основной целью выпускной квалификационной работы являлось исследование производственных систем с маршрутизацией, зависящей от состояния, разработка алгоритма метода анализа данных производственных систем, программная реализация алгоритма. Результатами работы являются следующие:

\begin{itemize}
\item рассмотрены производственные системы с маршрутизацией, зависящей от состояния;
\item приведено доказательство того, что при PSQ--маршрутизации марковский процесс обратим относительно времени и имеет мультипликативную форму стационарного распределения;
\item разработан алгоритм метода анализа производственных систем с маршрутизацией, зависящей от состояния;
\item разработана программа, вычисляющая основные стационарные характеристики;
\item проведены численные эксперименты с разработанной программой и приведены соответствующие результаты.
\end{itemize}

Эта модель может быть использована при решении задач анализа и оптимизации гибких производственных систем.


\end{document}
