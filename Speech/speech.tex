\documentclass[a4paper,14pt]{extarticle}

\usepackage{cmap} % improved search for russian words in pdf

% Nice cyrillic fonts
\usepackage{pscyr}
\renewcommand{\rmdefault}{ftm} % Times New Roman
%\renewcommand{\sfdefault}{ftx}
%\renewcommand{\ttdefault}{cmttp} % not good ttf

% Links
\usepackage{hyperref}
\hypersetup{
        unicode=true,
        colorlinks,
        citecolor=black,
        filecolor=black,
        linkcolor=black,
        urlcolor=blue
}
\usepackage[all]{hypcap} % fix links to floats

\usepackage{mathtext}
\usepackage[mathscr]{eucal}

\usepackage[T2A]{fontenc} % Russian support
\DeclareSymbolFont{T2Aletters}{T2A}{cmr}{m}{it}
\usepackage[utf8]{inputenc} % UTF8
\usepackage[english,russian]{babel}

% Mathematical AMS packages
\usepackage{amsmath, amsfonts, amsthm, amssymb, amscd}

% Provides support for setting the spacing between lines in a document. Package
% options include singlespacing, onehalfspacing, and doublespacing.
\usepackage{setspace}
\onehalfspacing % one half spacing

\usepackage{indentfirst} % indent
\setlength{\parindent}{1.25cm}

% Alternative geometry
\usepackage[top=2cm, bottom=2cm, left=2.5cm, right=1.5cm, bindingoffset=0cm,
 			headheight=0cm, footskip=1cm, headsep=0cm]{geometry}

% Nice citations [1,2,3,4] -> [1-4]
\usepackage[numbers,sort&compress]{natbib}

\usepackage{soul} % hyphenation for letterspacing, underlining and more

\sloppy % makes TeX less fussy about line breaking

% Support for the upright and bold greek letters
\usepackage{bm}
\usepackage[Symbolsmallscale]{upgreek}
\makeatletter
        \newcommand{\bfgreek}[1]{\bm{\@nameuse{up#1}}}
\makeatother

\usepackage{graphicx} % for graphics
\graphicspath{{images/}} % images path

\usepackage{tikz} % for drawing
\usetikzlibrary{shapes,arrows}

\usepackage{titlesec} % select alternative section titles
\usepackage{titletoc} % alternative headings for toc/lof/lot

% Keeps floats `in their place', preventing them from floating past a
% "\FloatBarrier" command into another section.  The floats should not move
% past every "\section".
\usepackage[section]{placeins}

\usepackage{longtable} % long table support
\usepackage{multirow,makecell,array}	 % advanced table style
\usepackage{tabularx}

\usepackage{float}

% Useful for individually placing figures on a separate page with
% \afterpage{\clearpage \begin{figure}[p] ... }
\usepackage{afterpage}


% --------------------------------------------------------------------------%
% Numeration of pages
% --------------------------------------------------------------------------%
\pagestyle{headings}
\makeatletter
\renewcommand{\@oddhead}{}
\renewcommand{\@oddfoot}{\hfil \thepage}
\setcounter{tocdepth}{2}
% --------------------------------------------------------------------------%


% --------------------------------------------------------------------------%
% Sections, subsections
% --------------------------------------------------------------------------%
% Numbering
\renewcommand{\thesection}{\arabic{section}}
\renewcommand{\thesubsection}{\arabic{section}.\arabic{subsection}}
\renewcommand{\thesubsubsection}
        {\arabic{section}.\arabic{subsection}.\arabic{subsubsection}}

\newcommand{\sectionbreak}{\clearpage}

% Contents, intro, conclusion
\newcommand{\structformat}
{
   \titleformat{\section}[block]
   {\centering\bfseries}
   {\thesection }{}{}
}

% Sections, subsections
\newcommand{\secformat}
{
    \titleformat{\section}[block]
    {\hspace{1.25cm}\raggedright\bfseries}
    {\thesection}{1ex}{}
}

\newcommand{\starsection}[1]{
    \structformat
    \section*{#1}
    \addcontentsline{toc}{section}{#1}
    \setcounter{section}{0}
    \secformat
}

\newcommand{\intro}{\starsection{ВВЕДЕНИЕ}}
\newcommand{\conclusion}{\starsection{ЗАКЛЮЧЕНИЕ}}

\titleformat{\subsection}[block]{\hspace{1.25cm}\normalfont\raggedright\bfseries}
		{\thesubsection}{1ex}{}
\titleformat{\subsubsection}[block]{\hspace{1.25cm}\normalfont\raggedright}
		{\thesubsubsection}{1ex}{}

\titlespacing*{\section}{0pt}{3ex plus 1ex minus .2ex}{3ex plus.2ex}
\titlespacing*{\subsection}{0pt}{2ex plus 1ex minus .2ex}{.3ex plus.2ex}
\titlespacing*{\subsubsection}{0pt}{2ex plus 1ex minus .2ex}{.3ex plus.2ex}
% --------------------------------------------------------------------------%


% --------------------------------------------------------------------------%
% Table and figure captions
% --------------------------------------------------------------------------%
\usepackage{caption}
\def\CaptionName#1{\gdef\@captionname{#1}}
\newlength\tmp %10cm
\setlength{\tmp}{1ex}
\setlength{\belowcaptionskip}{1ex}
\setlength{\abovecaptionskip}{1ex}

\captionsetup[table]{name=Таблица, labelsep=endash, format=plain, justification=RaggedRight,
			singlelinecheck=false, font={small}, position=top}
\captionsetup[figure]{name=Рисунок, labelsep=endash, justification=centering,
			font={small}, skip=\abovecaptionskip, position=below}
% --------------------------------------------------------------------------%


% --------------------------------------------------------------------------%
% Table and figure numbering by sections
% --------------------------------------------------------------------------%
\renewcommand{\theequation}{\arabic{section}.\arabic{equation}}
\renewcommand{\thefigure}{\arabic{section}.\arabic{figure}}
\renewcommand{\thetable}{\arabic{section}.\arabic{table}}

\makeatletter
\@addtoreset{equation}{section} % Equation counter
\@addtoreset{figure}{section} % Figure counter
\@addtoreset{table}{section} % Table counter
\makeatother
% --------------------------------------------------------------------------%


% --------------------------------------------------------------------------%
% Theorem, proof, definition, lemma, proposition, corollary
% --------------------------------------------------------------------------%
\newtheoremstyle{note}  % name
     {3mm}              % Space above
     {3mm}              % Space below
     {}                 % Body font
     {\parindent}       % Indent amount (empty = no indent, \parindent = para indent)
     {\bfseries}        % Thm head font
     {.}                % Punctuation after thm head
     { }                % Space after thm head: " " = normal interword space; \newline = linebreak
     {}                 % Thm head spec (can be left empty, meaning 'normal')

\theoremstyle{note}

\newtheorem{definition}{Определение}
\newtheorem{theorem}{Теорема}
\newtheorem{lemma}{Лемма}
\newtheorem{proposition}{Предложение}
\newtheorem{corollary}{Следствие}

\renewcommand{\proof}{\textbf{Доказательство.}\ignorespaces{\pushQED{\qed}}}
% --------------------------------------------------------------------------%


% --------------------------------------------------------------------------%
% Enumerations
% --------------------------------------------------------------------------%
\makeatletter
\renewcommand\theenumi  {\@arabic\c@enumi}
\renewcommand\theenumii {\@asbuk\c@enumii}
\renewcommand\theenumiii{\@roman\c@enumiii}
\renewcommand\theenumiv {\@Asbuk\c@enumiv}
\newcommand\atheenumi{\@asbuk\c@enumi}
\newcommand\atheenumii{\@asbuk\c@enumii}
\renewcommand\labelenumi  {\theenumi.}
\renewcommand\labelenumii {\theenumii.}
\renewcommand\labelenumiii{\theenumiii.}
\renewcommand\labelenumiv {\theenumiv.}
\renewcommand\p@enumii  {\theenumi}
\renewcommand\p@enumiii {\theenumi.\theenumii}
\renewcommand\p@enumiv  {\p@enumiii.\theenumiii}
\renewcommand\labelitemi  {\normalfont\bfseries\textemdash}
\renewcommand\labelitemii {\normalfont\bfseries\textendash}
\renewcommand\labelitemiii{\textperiodcentered}
\renewcommand\labelitemiv {\textasteriskcentered}

\renewcommand{\@listI}{%
\leftmargin=52pt
\rightmargin=0pt
\labelsep=7pt
\labelwidth=20pt
\itemindent=0pt
\listparindent=0pt
\topsep=4pt plus 2pt minus 4pt
\partopsep=0pt plus 1pt minus 1pt
\parsep=0pt plus 1pt
\itemsep=\parsep}
\makeatother

% Compressed lists: compactitem etc.
\usepackage{paralist}

\usepackage{enumitem}
\setlist[itemize]{fullwidth, listparindent=\parindent}
\setlist[enumerate]{fullwidth, itemindent=\parindent, listparindent=\parindent}
% --------------------------------------------------------------------------%

\begin{document}

\textbf{Слайд 1. Титульный лист}

Тема моей работы: производственные системы с маршрутизацией, зависящей от состояния.

Практическое значение этого направления определяется широким использованием сетей массового обслуживания в качестве математических моделей гибких производственных систем, необходимостью исследования производственных систем и оптимизации.

\textbf{Слайд 2. Цели и задачи работы}

Целью работы является исследование и анализ производственных систем с маршрутизацией, зависящей от состояния. Основными задачами являются: разработка алгоритма метода анализа данных производственных систем, программная реализация алгоритма и проведение численных экспериментов с разработанной программой.

%%%%%%%%%%%%%%%%%%%%%%%%%%%%%%%%%%%%%%%%%%%%%%%%%%%%%%%%%%%%%%%%%%%%%%%%%%%%%%%

\textbf{Слайд 3. Гибкие производственные системы}

В работе рассматривается гибкая производственная система, основные компоненты которой следующие:
\begin{enumerate}
\item Множество рабочих станций $C_i$.

\item Каждая рабочая станция производит обработку деталей разных типов $t$. Пусть $I_t$~--- множество рабочих станций, на которых производится обработка деталей типа $t$.

\item На станции $C_i$ есть $\kappa_i$ параллельно работающих приборов и локальное хранилище, где детали могут находиться в ожидании обработки. Определим вектор числа приборов в рабочих станциях $\kappa$.

\item Существует система транспортировки, обозначаемая как станция $C_0$, которая состоит из центрального хранилища и $\kappa_0$ транспортеров (это могут быть, например, конвейеры), которые осуществляют транспортировку деталей из центрального хранилища на рабочие станции. Переход деталей между рабочими станциями напрямую запрещен.

\item Общее число деталей типа $t$ в производственной системе постоянно и равно $N_t$. Определим вектор числа деталей $\mathbf{N}$.

\item Максимальное число деталей типа $t$ на станции $C_i$ ограничены емкостью рабочей станции $s_{it}$. Определим матрицу емкостей рабочих станций $s$.

\item Длительность обработки детали типа $t$ на станции $C_i$, имеет экспоненциальное распределение с параметром $\mu_{it}$, то есть $\mu_{it}$~--- это интенсивность обработки детали типа $t$ на станции $C_i$.

\item Дисциплина обработки на всех станциях $C_i$~--- \textit{RANDOM} (то есть детали выбираются для обработки случайным образом).

\item Определим состояние ГПС как вектор $\overline{\eta}$.
\end{enumerate}

%%%%%%%%%%%%%%%%%%%%%%%%%%%%%%%%%%%%%%%%%%%%%%%%%%%%%%%%%%%%%%%%%%%%%%%%%%%%%%%

\textbf{Слайд 4. Гибкие производственные системы}

Схема гибкой производственной системы представлена на рисунке.

Как только обработанная деталь покидает рабочую станцию, другая деталь того же типа сразу же поступает в нее. Рабочие станции или приборы могут быть свободны (простаивать), но они никогда не блокируются, так как имеется механизм, который постоянно забирает из рабочих станций обработанные детали и доставляет их обратно в центральное хранилище (предполагается, что в центральном хранилище достаточно мест для размещения всех деталей в случае необходимости).

Данная гибкая производственная система описывается неоднородной замкнутой экспоненциальной сетью массового обслуживания.

%%%%%%%%%%%%%%%%%%%%%%%%%%%%%%%%%%%%%%%%%%%%%%%%%%%%%%%%%%%%%%%%%%%%%%%%%%%%%%%

\textbf{Слайд 5. PSQ-маршрутизация}

Производственная система имеет маршрутизацию в кратчайшую очередь (PSQ--маршрутизацию). Вероятности перехода требований класса $t$ из системы $C_0$ в систему $C_i$ зависят от числа требований класса $t$ в двух системах, и принимают форму, представленную в формуле (1).

Можно заметить следующие особенности этой схемы маршрутизации:
\begin{itemize}
\item маршрутные вероятности выше для систем с б\'{о}льшим числом свободных приборов;
\item требования класса $t$ никогда (т.е. с вероятностью ноль) не направляются в систему, в которой все места в очереди для ожидания требованиями этого класса заняты.
\end{itemize}

%%%%%%%%%%%%%%%%%%%%%%%%%%%%%%%%%%%%%%%%%%%%%%%%%%%%%%%%%%%%%%%%%%%%%%%%%%%%%%%

\textbf{Слайд 6. Стационарное решение}

Стационарное решение для СеМО как модели гибкой производственной системы можно обобщить следующим образом.

\begin{theorem}
Марковский процесс $\overline{\eta}(\tau)$, определенный в пространстве состояний $S$ и управляемый PSQ--маршрутизацией, как определено в~(1), является обратимым относительно времени и имеет мультипликативную форму стационарного распределения вероятностей, как показано в формуле (2).
\end{theorem}

%%%%%%%%%%%%%%%%%%%%%%%%%%%%%%%%%%%%%%%%%%%%%%%%%%%%%%%%%%%%%%%%%%%%%%%%%%%%%%%

\textbf{Слайд 7. Структура алгоритма}

Алгоритм метода анализа сети массового обслуживания с PSQ-маршрутизацией имеет следующую последовательность шагов.

На первом шаге работы алгоритма вводятся параметры сети массового обслуживания.

На шаге 2 положим $i = 1$.

На шаге 3 перестанавливаем текущую СМО с последней путем перестановки индексов.

%%%%%%%%%%%%%%%%%%%%%%%%%%%%%%%%%%%%%%%%%%%%%%%%%%%%%%%%%%%%%%%%%%%%%%%%%%%%%%%

\textbf{Слайд 8. Структура алгоритма}

На 4 шаге происходит вычисление стационарного распределения вероятностей состояний текущей системы.

На 5 шаге перестанавливаем обратно текущую СМО.

Если $i < L$, то положить $i=i+1$ и перейти на \textbf{шаг 3}, иначе перейти к \textbf{шагу 7}.

На 7 шаге происходит вычисление стационарных характеристик СеМО и далее вывод результатов.

%%%%%%%%%%%%%%%%%%%%%%%%%%%%%%%%%%%%%%%%%%%%%%%%%%%%%%%%%%%%%%%%%%%%%%%%%%%%%%%

\textbf{Слайд 9. Интерфейс программы}

Для анализа производственных систем с маршрутизацией, зависящей от состояния, была разработана программа.

Программа позволяет вычислить стационарное распределение и основные характеристики рассмотренных СеМО. Вычисления могут производится как для однородной, так и для неоднородной СеМО.

Разработанная программа имеет графический интерфейс. Входные данные считываются с формы или из файла, проверяются на корректность, и в соответствии с проверкой либо производится анализ, либо выдается сообщение об ошибке.

%%%%%%%%%%%%%%%%%%%%%%%%%%%%%%%%%%%%%%%%%%%%%%%%%%%%%%%%%%%%%%%%%%%%%%%%%%%%%%%

\textbf{Слайд 10. Интерфейс программы}

При нажатии кнопки "Получить результаты" открывается окно с подсчитанными в ходе анализа основными характеристиками СМО и стационарным распределением. Здесь существует возможность сохранить результаты анализа в файл.

%%%%%%%%%%%%%%%%%%%%%%%%%%%%%%%%%%%%%%%%%%%%%%%%%%%%%%%%%%%%%%%%%%%%%%%%%%%%%%%

\textbf{Слайд 11. Аспекты практического применения}

С помощью разработанной программы было проведено несколько экспериментов. Приведем один из них. Начальные данные представлены на слайде (...), ниже в таблице приведены полученные результаты анализа.

%%%%%%%%%%%%%%%%%%%%%%%%%%%%%%%%%%%%%%%%%%%%%%%%%%%%%%%%%%%%%%%%%%%%%%%%%%%%%%%

\textbf{Слайд 12. Результаты работы}

Результаты работы следующие:

\begin{itemize}
\item рассмотрены производственные системы с маршрутизацией, зависящей от состояния;
\item приведена теорема, согласно которой соответствующая СеМО имеет стационарное распределение;
\item разработан алгоритм метода анализа данных производственных систем;
\item разработана программа, вычисляющая основные стационарные характеристики;
\item проведены численные эксперименты с разработанной программой и приведены соответствующие результаты.
\end{itemize}

\end{document}
