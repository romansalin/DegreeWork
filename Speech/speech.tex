\documentclass[a4paper,14pt]{extarticle}

\usepackage{cmap} % improved search for russian words in pdf

% Nice cyrillic fonts
\usepackage{pscyr}
\renewcommand{\rmdefault}{ftm} % Times New Roman
%\renewcommand{\sfdefault}{ftx}
%\renewcommand{\ttdefault}{cmttp} % not good ttf

% Links
\usepackage{hyperref}
\hypersetup{
        unicode=true,
        colorlinks,
        citecolor=black,
        filecolor=black,
        linkcolor=black,
        urlcolor=blue
}
\usepackage[all]{hypcap} % fix links to floats

\usepackage{mathtext}
\usepackage[mathscr]{eucal}

\usepackage[T2A]{fontenc} % Russian support
\DeclareSymbolFont{T2Aletters}{T2A}{cmr}{m}{it}
\usepackage[utf8]{inputenc} % UTF8
\usepackage[english,russian]{babel}

% Mathematical AMS packages
\usepackage{amsmath, amsfonts, amsthm, amssymb, amscd}

% Provides support for setting the spacing between lines in a document. Package
% options include singlespacing, onehalfspacing, and doublespacing.
\usepackage{setspace}
\onehalfspacing % one half spacing

\usepackage{indentfirst} % indent
\setlength{\parindent}{1.25cm}

% Alternative geometry
\usepackage[top=2cm, bottom=2cm, left=2.5cm, right=1.5cm, bindingoffset=0cm,
 			headheight=0cm, footskip=1cm, headsep=0cm]{geometry}

% Nice citations [1,2,3,4] -> [1-4]
\usepackage[numbers,sort&compress]{natbib}

\usepackage{soul} % hyphenation for letterspacing, underlining and more

\sloppy % makes TeX less fussy about line breaking

% Support for the upright and bold greek letters
\usepackage{bm}
\usepackage[Symbolsmallscale]{upgreek}
\makeatletter
        \newcommand{\bfgreek}[1]{\bm{\@nameuse{up#1}}}
\makeatother

\usepackage{graphicx} % for graphics
\graphicspath{{images/}} % images path

\usepackage{tikz} % for drawing
\usetikzlibrary{shapes,arrows}

\usepackage{titlesec} % select alternative section titles
\usepackage{titletoc} % alternative headings for toc/lof/lot

% Keeps floats `in their place', preventing them from floating past a
% "\FloatBarrier" command into another section.  The floats should not move
% past every "\section".
\usepackage[section]{placeins}

\usepackage{longtable} % long table support
\usepackage{multirow,makecell,array}	 % advanced table style
\usepackage{tabularx}

\usepackage{float}

% Useful for individually placing figures on a separate page with
% \afterpage{\clearpage \begin{figure}[p] ... }
\usepackage{afterpage}


% --------------------------------------------------------------------------%
% Numeration of pages
% --------------------------------------------------------------------------%
\pagestyle{headings}
\makeatletter
\renewcommand{\@oddhead}{}
\renewcommand{\@oddfoot}{\hfil \thepage}
\setcounter{tocdepth}{2}
% --------------------------------------------------------------------------%


% --------------------------------------------------------------------------%
% Sections, subsections
% --------------------------------------------------------------------------%
% Numbering
\renewcommand{\thesection}{\arabic{section}}
\renewcommand{\thesubsection}{\arabic{section}.\arabic{subsection}}
\renewcommand{\thesubsubsection}
        {\arabic{section}.\arabic{subsection}.\arabic{subsubsection}}

\newcommand{\sectionbreak}{\clearpage}

% Contents, intro, conclusion
\newcommand{\structformat}
{
   \titleformat{\section}[block]
   {\centering\bfseries}
   {\thesection }{}{}
}

% Sections, subsections
\newcommand{\secformat}
{
    \titleformat{\section}[block]
    {\hspace{1.25cm}\raggedright\bfseries}
    {\thesection}{1ex}{}
}

\newcommand{\starsection}[1]{
    \structformat
    \section*{#1}
    \addcontentsline{toc}{section}{#1}
    \setcounter{section}{0}
    \secformat
}

\newcommand{\intro}{\starsection{ВВЕДЕНИЕ}}
\newcommand{\conclusion}{\starsection{ЗАКЛЮЧЕНИЕ}}

\titleformat{\subsection}[block]{\hspace{1.25cm}\normalfont\raggedright\bfseries}
		{\thesubsection}{1ex}{}
\titleformat{\subsubsection}[block]{\hspace{1.25cm}\normalfont\raggedright}
		{\thesubsubsection}{1ex}{}

\titlespacing*{\section}{0pt}{3ex plus 1ex minus .2ex}{3ex plus.2ex}
\titlespacing*{\subsection}{0pt}{2ex plus 1ex minus .2ex}{.3ex plus.2ex}
\titlespacing*{\subsubsection}{0pt}{2ex plus 1ex minus .2ex}{.3ex plus.2ex}
% --------------------------------------------------------------------------%


% --------------------------------------------------------------------------%
% Table and figure captions
% --------------------------------------------------------------------------%
\usepackage{caption}
\def\CaptionName#1{\gdef\@captionname{#1}}
\newlength\tmp %10cm
\setlength{\tmp}{1ex}
\setlength{\belowcaptionskip}{1ex}
\setlength{\abovecaptionskip}{1ex}

\captionsetup[table]{name=Таблица, labelsep=endash, format=plain, justification=RaggedRight,
			singlelinecheck=false, font={small}, position=top}
\captionsetup[figure]{name=Рисунок, labelsep=endash, justification=centering,
			font={small}, skip=\abovecaptionskip, position=below}
% --------------------------------------------------------------------------%


% --------------------------------------------------------------------------%
% Table and figure numbering by sections
% --------------------------------------------------------------------------%
\renewcommand{\theequation}{\arabic{section}.\arabic{equation}}
\renewcommand{\thefigure}{\arabic{section}.\arabic{figure}}
\renewcommand{\thetable}{\arabic{section}.\arabic{table}}

\makeatletter
\@addtoreset{equation}{section} % Equation counter
\@addtoreset{figure}{section} % Figure counter
\@addtoreset{table}{section} % Table counter
\makeatother
% --------------------------------------------------------------------------%


% --------------------------------------------------------------------------%
% Theorem, proof, definition, lemma, proposition, corollary
% --------------------------------------------------------------------------%
\newtheoremstyle{note}  % name
     {3mm}              % Space above
     {3mm}              % Space below
     {}                 % Body font
     {\parindent}       % Indent amount (empty = no indent, \parindent = para indent)
     {\bfseries}        % Thm head font
     {.}                % Punctuation after thm head
     { }                % Space after thm head: " " = normal interword space; \newline = linebreak
     {}                 % Thm head spec (can be left empty, meaning 'normal')

\theoremstyle{note}

\newtheorem{definition}{Определение}
\newtheorem{theorem}{Теорема}
\newtheorem{lemma}{Лемма}
\newtheorem{proposition}{Предложение}
\newtheorem{corollary}{Следствие}

\renewcommand{\proof}{\textbf{Доказательство.}\ignorespaces{\pushQED{\qed}}}
% --------------------------------------------------------------------------%


% --------------------------------------------------------------------------%
% Enumerations
% --------------------------------------------------------------------------%
\makeatletter
\renewcommand\theenumi  {\@arabic\c@enumi}
\renewcommand\theenumii {\@asbuk\c@enumii}
\renewcommand\theenumiii{\@roman\c@enumiii}
\renewcommand\theenumiv {\@Asbuk\c@enumiv}
\newcommand\atheenumi{\@asbuk\c@enumi}
\newcommand\atheenumii{\@asbuk\c@enumii}
\renewcommand\labelenumi  {\theenumi.}
\renewcommand\labelenumii {\theenumii.}
\renewcommand\labelenumiii{\theenumiii.}
\renewcommand\labelenumiv {\theenumiv.}
\renewcommand\p@enumii  {\theenumi}
\renewcommand\p@enumiii {\theenumi.\theenumii}
\renewcommand\p@enumiv  {\p@enumiii.\theenumiii}
\renewcommand\labelitemi  {\normalfont\bfseries\textemdash}
\renewcommand\labelitemii {\normalfont\bfseries\textendash}
\renewcommand\labelitemiii{\textperiodcentered}
\renewcommand\labelitemiv {\textasteriskcentered}

\renewcommand{\@listI}{%
\leftmargin=52pt
\rightmargin=0pt
\labelsep=7pt
\labelwidth=20pt
\itemindent=0pt
\listparindent=0pt
\topsep=4pt plus 2pt minus 4pt
\partopsep=0pt plus 1pt minus 1pt
\parsep=0pt plus 1pt
\itemsep=\parsep}
\makeatother

% Compressed lists: compactitem etc.
\usepackage{paralist}

\usepackage{enumitem}
\setlist[itemize]{fullwidth, listparindent=\parindent}
\setlist[enumerate]{fullwidth, itemindent=\parindent, listparindent=\parindent}
% --------------------------------------------------------------------------%

\begin{document}

\intro

Тема моей дипломной работы: производственные системы с маршрутизацией, зависящей от состояния. В моей дипломной работе рассматривается сеть массового обслуживания с зависимой от состояния маршрутизацией как модель гибкой производственной системы.

Целью дипломной работы является исследование и анализ производственных систем с маршрутизацией, зависящей от состояния. Задачами являются: разработка алгоритма метода анализа данных производственных систем, программная реализация алгоритма и проведение численных экспериментов с разработанной программой.

Практическое значение этого направления определяется широким использованием сетей массового обслуживания в качестве математических моделей гибких производственных систем, необходимостью исследования производственных систем и их оптимизации.

%%%%%%%%%%%%%%%%%%%%%%%%%%%%%%%%%%%%%%%%%%%%%%%%%%%%%%%%%%%%%%%%%%%%%%%%%%%%%%%


\section{Гибкие производственные системы с маршрутизацией, зависящей от состояния}
\label{sec:FMS}

\subsection{Описание модели}
\label{subsec:FMS_specification}

Рассмотрим гибкую производственную систему, основные компоненты которой следующие.

\begin{enumerate}
\item Множество рабочих станций (систем) $C_i$ с номерами из множества $I \equiv \{ i ~|~ i=1,...,L \}$. Каждая рабочая станция $C_i$ производит обработку деталей и может выполнять один или несколько типов производственных операций $t$ (например, сверление, фрезерование, и т.д.). На станции $C_i$ есть $\kappa_i$ параллельно работающих машин. Максимальное число деталей, которое допускается в любой момент времени (включая как обрабатывающиеся детали, так и детали, ожидающие в локальном хранилище), ограничены $s_i$ (емкость рабочей станции), где $s_i \geqslant \kappa_i, ~ i=1,...,L$. Определим вектор числа приборов в рабочих станциях $\kappa = (\kappa_i)$ и вектор емкостей рабочих станций $s = (s_i)$, $i=1,...,L$.

\item Система транспортировки материалов (MHS), обозначаемая как станция $C_0$, которая состоит из $\kappa_0$ транспортеров (например, конвейеры), которые осуществляют транспортировку деталей между рабочими станциями, и центрального хранилища.

\item Палеты, на которых перемещаются детали. Для каждой детали выделяется одна палета. Общее количество палет, доступных в ГПС, постоянно и равно $N$.
\end{enumerate}

Схема гибкой производственной системы представлена на рисунке.

Функционирование гибкой производственной системы происходит следующим образом.

Система должна обработать $t=1,2,...,T$ типов деталей. Общее число деталей в системе в любой момент времени постоянно и равно $N$. Всякий раз, когда обработанная деталь покидает систему, другая деталь того же типа сразу же поступает в систему. Число палет $N_t$, выделяемых для деталей типа $t$, постоянно и $\sum\limits_t N_t = N$. Определим вектор начального числа деталей $\mathbf{N}=(N_t)$, $t=1,...,T$.

Рабочие станции или машины могут быть свободны (простаивать), но они никогда не блокируются. На практике это, как правило, обеспечивается за счет \textit{возвратного конвейера}, который постоянно забирает из рабочих станций детали, завершившие обработку, и доставляет их обратно в центральное хранилище (предполагается, что в центральном хранилище достаточно мест для размещения всех деталей в случае необходимости).

Деталь каждого типа $t$ нуждается во множестве операций, которые будут выполняться на наборе рабочих станций с номерами $I_t \subseteq I$. Длительность обработки детали типа $t$ на станции $C_i$, $i \in I_t$, имеет экспоненциальное распределение с параметром $\mu_{it}$. $s_{it}$~--- емкость хранилища, выделенного для деталей типа $t$ на станции $C_i$.

Дисциплина обработки на всех станциях $C_i,~i=0,...,L$,~--- \textit{RANDOM} (то есть детали выбираются для обработки случайным образом). 

Пусть $\Theta = (\theta_{it,jt})$~--- маршрутная матрица, $t=1,...,T$, $i,j=0,...,L$, где $\theta_{it,jt}$~--- вероятность того, что деталь типа $t$ после обработки на станции $C_i$ поступает на станцию $C_j$. Заметим, что допускаются переходы только из системы транспортировки на рабочие станции и из рабочих станций в систему транспортировки, т.е. $\theta_{it,jt}=0$.

Данная гибкая производственная система с введенными выше предположениями описывается неоднородной замкнутой экспоненциальной сетью массового обслуживания $\Gamma=\left<L,T,\mathbf{N},N,M,\Theta,\kappa,\mu,\textit{RANDOM}\right>$.

\subsection{Решение уравнения равновесия}
\label{subsec:solution}

Определим состояние сети $\Gamma$, соответствующей рассмотренной выше гибкой производственной системе. $\{ \overline{\eta}(\tau) \}$ определяет процесс Маркова со следующим конечным пространством состояний: (...).

Сформулируем PSQ--маршрутизацию следующим образом. Вероятности перехода требований класса $t$ из системы $C_0$ в систему $C_i$, $i \in I_t$, зависят от $n_{0t}$ и от $n_{it}$~--- числа требований класса $t$ в двух системах, и принимают форму

Несложно заметить следующие особенности этой схемы маршрутизации:
\begin{itemize}
\item маршрутные вероятности выше для систем с б\'{о}льшим числом свободных приборов;
\item требования класса $t$ никогда (т.е. с вероятностью ноль) не направляются в систему, в которой все места в очереди для ожидания требованиями этого класса заняты.
\end{itemize}

Стационарное решение для сети $\Gamma$ как модели гибкой производственной системы можно теперь обобщить следующим образом.

\begin{theorem}
 Марковский процесс $\overline{\eta}(\tau)$, определенный в пространстве состояний $S$ и управляемый PSQ--маршрутизацией, как определено в~(\ref{eq:2.2}), является обратимым относительно времени и имеет следующую мультипликативную форму стационарного распределения вероятностей: (...).
\end{theorem}


%%%%%%%%%%%%%%%%%%%%%%%%%%%%%%%%%%%%%%%%%%%%%%%%%%%%%%%%%%%%%%%%%%%%%%%%%%%%%%%


\section{Алгоритм метода анализа производственных систем с~маршрутизацией, зависящей от состояния}
\label{sec:algorithm}

\subsection{Описание алгоритма}
\label{subsec:algorithm_description}

Рассмотрим неоднородную замкнутую экспоненциальную сеть массового обслуживания $L$ системами (включая систему транспортировки материалов, которая теперь обозначается как любая $C_i$) и $N$ требованиями.

Не теряя общности, предположим, что требуется получить предельное частное распределение вероятностей для системы $C_L$ сети $\Gamma'$.

Определим векторы размерности $T$: (...). Пусть $G(m,\mathbf{N})$ --- нормализующая константа. Кроме того, определим (...).
Обозначим через $\pi_m(\mathbf{n},\mathbf{N})$ предельную вероятность того, что общее число требований в системе $m$ будет равно $\mathbf{n}$, когда общее число требований всей сети равно $\mathbf{N}$. Тогда справедливо следующее следствие (...).

\subsection{Структурная схема алгоритма}
\label{subsec:flowchart}

Алгоритм метода анализа сети массового обслуживания $\Gamma$ с маршрутизацией, зависящей от состояния, имеет блочную структуру, представленную на рисунке (...).

\medskip
\textbf{Блок 1. Ввод исходных данных}

На начальном этапе работы алгоритма вводятся параметры сети массового обслуживания $\Gamma$:\\
$L$~--- число СМО в СеМО;\\
$\mathbf{N}=(N_t)$~--- вектор начального числа требований в СеМО, $t=1,...,T$;\\
$\kappa=(\kappa_i)$~--- вектор числа приборов в системах обслуживания СеМО, $i=0,...,L$;\\
$s=(s_{it})$~--- матрица емкостей систем в СеМО, $i=0,...,L,~t=1,...,T$;\\
$\mu=(\mu_{it})$~--- матрица интенсивностей обслуживания требований системами СеМО, $i=0,...,L,~t=1,...,T$.

\medskip
\textbf{Блок 2. Перестановка СМО $\boldsymbol{C_i}$ и $\boldsymbol{C_L}$}

Во втором блоке для вычисления стационарного распределения вероятностей состояний системы $C_i,~i=1,...,L,$ происходит перестановка системы $C_i$ с последней системой $C_L$. Это осуществляется путем перестановки соответствующих элементов векторов.

\medskip
\textbf{Блок 3. Вычисление стационарного распределения вероятностей состояний СМО $\boldsymbol{C_i}$}

% @TODO что-то придумать

\medskip
\textbf{Блок 5. Вычисление стационарных характеристик СеМО}

На данном этапе происходит вычисление следующих стационарных характеристик СеМО:
\begin{itemize}
\item м. о. числа $t$-требований в СМО;
\item м. о. числа занятых $t$-требованиями приборов в СМО;
\item интенсивность входящего потока $t$-требований в СМО;
\item коэффициенты использования обслуживающих приборов СМО $t$~-~требованиями.
\end{itemize}
Эти характеристики вычисляются по формулам (...).

\medskip
\textbf{Блок 6. Вывод результатов}

В данном блоке происходит вывод (на экран или в файл) стационарного распределения и стационарных характеристик, полученных в блоке 4 и 5.


%%%%%%%%%%%%%%%%%%%%%%%%%%%%%%%%%%%%%%%%%%%%%%%%%%%%%%%%%%%%%%%%%%%%%%%%%%%%%%%


\section{Описание и назначение программы}
\label{sec:program_description_and_purpose}

Была разработана программа на языке Java, предназначенная для анализа производственных систем с маршрутизацией, зависящей от состояния.

Программа позволяет вычислить стационарное распределение и основные характеристики производственных систем с маршрутизацией, зависящей от состояния. Вычисления могут производится как для однородной, так и для неоднородной сети массового обслуживания.

Разработанная программа имеет графический интерфейс. Входные данные считываются с формы, проверяются на корректность, и в соответствии с проверкой либо производится анализ, либо выдается сообщение об ошибке. Для рассматриваемой сети входными данными являются: число систем, число классов требований, вектор числа требований определенного класса, вектор числа обслуживающих приборов, емкости систем и интенсивности обслуживания. Выходными данными являются стационарное распределение и основные стационарные характеристики систем, а именно: м. о. числа требований, интенсивности потока требований и коэффициенты использования обслуживающих приборов.

При запуске программы появляется окно, изображенное на рисунке (...). Для удобства существует возможность открыть файл с заданными в нем входными данными.

Для анализа при заданных входных данных служит кнопка <<Получить результаты>>. При ее нажатии открывается окно с подсчитанными в ходе анализа основными характеристиками СМО и стационарным распределением (слайд ...).


%%%%%%%%%%%%%%%%%%%%%%%%%%%%%%%%%%%%%%%%%%%%%%%%%%%%%%%%%%%%%%%%%%%%%%%%%%%%%%%


\section{Аспекты практического применения}
\label{sec:practical_application}

Было проведено несколько серий экспериментов с использованием разработанной программы анализа производственных систем с маршрутизацией, зависящей от состояния.

\textbf{Эксперимент 1}

Рассмотрим производственную систему с $18$ машинами, которые сгруппированы по $9$ рабочим станциям. Число приборов, интенсивность обработки детали одним прибором и емкость локального хранилища на каждой станции соответственно равны: (...)

Математическое ожидание числа деталей ($\overline{n}_i$), пропускная способность ($\lambda_i$) и коэффициенты использования приборов ($\psi_i$) для каждой станции приведены в таблице (...).

\textbf{Эксперимент 2}

Возьмем гибкую производственную систему из примера 3 и посмотрим, как будут изменяться характеристики рабочих станций $C_i$, $i=1,2$, с изменением числа приборов, емкостей рабочих станций и интенсивностей обработки деталей (эти данные предполагаются одинаковыми для обеих рабочих станций). Результаты представлены на рисунках (...).


%%%%%%%%%%%%%%%%%%%%%%%%%%%%%%%%%%%%%%%%%%%%%%%%%%%%%%%%%%%%%%%%%%%%%%%%%%%%%%%


\conclusion
Результаты дипломной работы следующие:

\begin{itemize}
\item рассмотрены производственные системы с маршрутизацией, зависящей от состояния;
\item приведено доказательство того, что при PSQ--маршрутизации марковский процесс обратим относительно времени и имеет мультипликативную форму стационарного распределения;
\item разработан алгоритм метода анализа производственных систем с маршрутизацией, зависящей от состояния;
\item разработана программа, вычисляющая основные стационарные характеристики;
\item проведены численные эксперименты с разработанной программой и приведены соответствующие результаты.
\end{itemize}

Эта модель может быть использована при решении задач анализа и оптимизации гибких производственных систем.


\end{document}
